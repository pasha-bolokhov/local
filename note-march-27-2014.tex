\documentclass[epsfig,12pt]{article}
\usepackage{epsfig}
\usepackage{graphicx}
\usepackage{rotating}
\usepackage{latexsym}
\usepackage{amsmath}
\usepackage{amssymb}
\usepackage{relsize}
\usepackage{geometry}
\geometry{letterpaper}
\usepackage{color}
\usepackage{bm}
\usepackage{slashed}
%\usepackage{showlabels}




%%%%%%%%%%%%%%%%%%%%%%%%%%%%%%%%%%%%%%%%%%%%%%%%%%%%%%%%%%%%%%%%%%%%%%%%%%%%%%%%
%                                                                              %
%                                                                              %
%                     D O C U M E N T   S E T T I N G S                        %
%                                                                              %
%                                                                              %
%%%%%%%%%%%%%%%%%%%%%%%%%%%%%%%%%%%%%%%%%%%%%%%%%%%%%%%%%%%%%%%%%%%%%%%%%%%%%%%%
\def\baselinestretch{1.1}

\hyphenation{con-fi-ning}
\hyphenation{Cou-lomb}
\hyphenation{Yan-ki-e-lo-wicz}




%%%%%%%%%%%%%%%%%%%%%%%%%%%%%%%%%%%%%%%%%%%%%%%%%%%%%%%%%%%%%%%%%%%%%%%%%%%%%%%%
%                                                                              %
%                                                                              %
%                      C O M M O N   D E F I N I T I O N S                     %
%                                                                              %
%                                                                              %
%%%%%%%%%%%%%%%%%%%%%%%%%%%%%%%%%%%%%%%%%%%%%%%%%%%%%%%%%%%%%%%%%%%%%%%%%%%%%%%%
\def\beq{\begin{equation}}
\def\eeq{\end{equation}}
\def\beqn{\begin{eqnarray}}
\def\eeqn{\end{eqnarray}}
\def\beqn{\begin{eqnarray}}
\def\eeqn{\end{eqnarray}}
\def\nn{\nonumber}
\def\ba{\beq\new\begin{array}{c}}
\def\ea{\end{array}\eeq}
\def\be{\ba}
\def\ee{\ea}


\newcommand{\nfour}{${\cal N}=4\;$}
\newcommand{\none}{${\mathcal N}=1\,$}
\newcommand{\nonen}{${\mathcal N}=1$}
\newcommand{\ntwo}{${\mathcal N}=2$}
\newcommand{\ntt}{${\mathcal N}=(2,2)\,$}
\newcommand{\nzt}{${\mathcal N}=(0,2)\,$}
\newcommand{\ntwon}{${\mathcal N}=2$}
\newcommand{\ntwot}{${\mathcal N}= \left(2,2\right) $ }
\newcommand{\ntwoo}{${\mathcal N}= \left(0,2\right) $ }
\newcommand{\ntwoon}{${\mathcal N}= \left(0,2\right)$}


\newcommand{\ca}{{\mathcal A}}
\newcommand{\cell}{{\mathcal L}}
\newcommand{\cw}{{\mathcal W}}
\newcommand{\cs}{{\mathcal S}}
\newcommand{\vp}{\varphi}
\newcommand{\pt}{\partial}
\newcommand{\ve}{\varepsilon}
\newcommand{\gs}{g^{2}}
\newcommand{\zn}{$Z_N$}
\newcommand{\cd}{${\mathcal D}$}
\newcommand{\cde}{{\mathcal D}}
\newcommand{\cf}{${\mathcal F}$}
\newcommand{\cfe}{{\mathcal F}}
\newcommand{\ff}{\mc{F}}
\newcommand{\bff}{\ov{\mc{F}}}


\newcommand{\p}{\partial}
\newcommand{\wt}{\widetilde}
\newcommand{\ov}{\overline}
\newcommand{\mc}[1]{\mathcal{#1}}
\newcommand{\md}{\mathcal{D}}
\newcommand{\ml}{\mathcal{L}}
\newcommand{\mw}{\mathcal{W}}
\newcommand{\ma}{\mathcal{A}}


\newcommand{\GeV}{{\rm GeV}}
\newcommand{\eV}{{\rm eV}}
\newcommand{\Heff}{{\mathcal{H}_{\rm eff}}}
\newcommand{\Leff}{{\mathcal{L}_{\rm eff}}}
\newcommand{\el}{{\rm EM}}
\newcommand{\uflavor}{\mathbf{1}_{\rm flavor}}
\newcommand{\lgr}{\left\lgroup}
\newcommand{\rgr}{\right\rgroup}


\newcommand{\Mpl}{M_{\rm Pl}}
\newcommand{\suc}{{{\rm SU}_{\rm C}(3)}}
\newcommand{\sul}{{{\rm SU}_{\rm L}(2)}}
\newcommand{\sutw}{{\rm SU}(2)}
\newcommand{\suth}{{\rm SU}(3)}
\newcommand{\ue}{{\rm U}(1)}


\newcommand{\LN}{\Lambda_\text{SU($N$)}}
\newcommand{\sunu}{{\rm SU($N$) $\times$ U(1) }}
\newcommand{\sunun}{{\rm SU($N$) $\times$ U(1)}}
\def\cfl {$\text{SU($N$)}_{\rm C+F}$ }
\def\cfln {$\text{SU($N$)}_{\rm C+F}$}
\newcommand{\mUp}{m_{\rm U(1)}^{+}}
\newcommand{\mUm}{m_{\rm U(1)}^{-}}
\newcommand{\mNp}{m_\text{SU($N$)}^{+}}
\newcommand{\mNm}{m_\text{SU($N$)}^{-}}
\newcommand{\AU}{\mc{A}^{\rm U(1)}}
\newcommand{\AN}{\mc{A}^\text{SU($N$)}}
\newcommand{\aU}{a^{\rm U(1)}}
\newcommand{\aN}{a^\text{SU($N$)}}
\newcommand{\baU}{\ov{a}{}^{\rm U(1)}}
\newcommand{\baN}{\ov{a}{}^\text{SU($N$)}}
\newcommand{\lU}{\lambda^{\rm U(1)}}
\newcommand{\lN}{\lambda^\text{SU($N$)}}
\newcommand{\bxir}{\ov{\xi}{}_R}
\newcommand{\bxil}{\ov{\xi}{}_L}
\newcommand{\xir}{\xi_R}
\newcommand{\xil}{\xi_L}
\newcommand{\bzl}{\ov{\zeta}{}_L}
\newcommand{\bzr}{\ov{\zeta}{}_R}
\newcommand{\zr}{\zeta_R}
\newcommand{\zl}{\zeta_L}
\newcommand{\nbar}{\ov{n}}
\newcommand{\nnbar}{n\ov{n}}
\newcommand{\muU}{\mu_\text{U}}


\newcommand{\cpn}{CP$^{N-1}$\,}
\newcommand{\CPC}{CP($N-1$)$\times$C }
\newcommand{\CPCn}{CP($N-1$)$\times$C}


\newcommand{\lar}{\lambda_R}
\newcommand{\lal}{\lambda_L}
\newcommand{\larl}{\lambda_{R,L}}
\newcommand{\lalr}{\lambda_{L,R}}
\newcommand{\blar}{\ov{\lambda}{}_R}
\newcommand{\blal}{\ov{\lambda}{}_L}
\newcommand{\blarl}{\ov{\lambda}{}_{R,L}}
\newcommand{\blalr}{\ov{\lambda}{}_{L,R}}


\newcommand{\bgamma}{\ov{\gamma}}
\newcommand{\bpsi}{\ov{\psi}{}}
\newcommand{\bphi}{\ov{\phi}{}}
\newcommand{\bxi}{\ov{\xi}{}}


\newcommand{\qt}{\wt{q}}
\newcommand{\bq}{\ov{q}}
\newcommand{\bqt}{\overline{\widetilde{q}}}


\newcommand{\eer}{\epsilon_R}
\newcommand{\eel}{\epsilon_L}
\newcommand{\eerl}{\epsilon_{R,L}}
\newcommand{\eelr}{\epsilon_{L,R}}
\newcommand{\beer}{\ov{\epsilon}{}_R}
\newcommand{\beel}{\ov{\epsilon}{}_L}
\newcommand{\beerl}{\ov{\epsilon}{}_{R,L}}
\newcommand{\beelr}{\ov{\epsilon}{}_{L,R}}


\newcommand{\bi}{{\bar \imath}}
\newcommand{\bj}{{\bar \jmath}}
\newcommand{\bk}{{\bar k}}
\newcommand{\bl}{{\bar l}}
\newcommand{\bmm}{{\bar m}}


\newcommand{\nz}{{n^{(0)}}}
\newcommand{\no}{{n^{(1)}}}
\newcommand{\bnz}{{\ov{n}{}^{(0)}}}
\newcommand{\bno}{{\ov{n}{}^{(1)}}}
\newcommand{\Dz}{{D^{(0)}}}
\newcommand{\Do}{{D^{(1)}}}
\newcommand{\bDz}{{\ov{D}{}^{(0)}}}
\newcommand{\bDo}{{\ov{D}{}^{(1)}}}
\newcommand{\sigz}{{\sigma^{(0)}}}
\newcommand{\sigo}{{\sigma^{(1)}}}
\newcommand{\bsigz}{{\ov{\sigma}{}^{(0)}}}
\newcommand{\bsigo}{{\ov{\sigma}{}^{(1)}}}


\newcommand{\rrenz}{{r_\text{ren}^{(0)}}}
\newcommand{\bren}{{\beta_\text{ren}}}


\newcommand{\Tr}{\text{Tr}}
\newcommand{\Ts}{\text{Ts}}
\newcommand{\dm}{\hat{{\scriptstyle \Delta} m}}
\newcommand{\dmdag}{\hat{{\scriptstyle \Delta} m}{}^\dag}
\newcommand{\mhat}{\widehat{m}}
\newcommand{\deltam}{{\scriptstyle \Delta} m}
\newcommand{\nvac}{\vec{n}{}_\text{vac}}


\newcommand{\ie}{{\it i.e.}~}
\newcommand{\eg}{{\it e.g.}~}
\newcommand{\ansatz}{{\it ansatz} }


\begin{document}


	When considering duality, it is natural to think of the fieldstrength rather than of the gauge potential.
	Faddeev used a trick to introduce fieldstrength variables in his book on {\it Gauge Fields} when he
	described a way to quantize the Yang-Mills theory, 
\beq
\label{QED}
	\cell    ~~=~~    \frac{1}{2}\, \p_{[\mu} A_{\nu]}\, F_{\mu\nu}  ~~-~~  \frac{1}{4}\, F_{\mu\nu}^2  ~~+~~ j_\mu A^\mu\,.
\eeq
	Here $ F_{\mu\nu} $ are auxiliary variables, which, upon using their equations of motion resolve to
	the usual expression 
\[
	F_{\mu\nu} \,=\, \p_{[\mu}A_{\nu]}\,.
\]
	If substituted back, they turn the Lagrangian into the canonical form.
	
	In order to write a mutually-dual theory, we need to introduce two gauge fields --- and accordingly, two fieldstrengths
\begin{align}
%
\notag
	\cell    & ~~=~~    \frac{1}{2}\, \p_{[\mu} A_{\nu]}\, F_{\mu\nu}  ~~-~~  \frac{1}{4}\, F_{\mu\nu}^2  ~~+~~ j_\mu\, A^\mu  ~~+~~
	\\[2mm]
%
	&
		   ~~+~~    \frac{1}{2}\, \p_{[\mu} B_{\nu]}\, G_{\mu\nu}  ~~-~~  \frac{1}{4}\, G_{\mu\nu}^2  ~~+~~ k_\mu\, B^\mu\,.
\end{align}
	To make these gauge fields dual to each other, one needs to impose the corresponding duality constraint.
	Again, it is easier to do in terms of $ F_{\mu\nu} $ and $ G_{\mu\nu} $ than in terms of the gauge potentials.
	It seems that a constraint like
\[
	\lambda\, \big( F_{\mu\nu} ~-~ \wt G{}_{\mu\nu} \big)^2
	\qquad\qquad \text{or, equivalently,} \qquad\qquad
	C_{\mu\nu}\, \big( F^{\mu\nu} ~-~ \wt G{}^{\mu\nu} \big)
\]
	(the latter could better suit Minkowski space) could be a natural choice,
	as this is indeed a way of saying that the magnetic field of one field is the electric field of the other,
	and {\it vice versa}.
	Quite surprisingly, this constraint does not work, and leads to a theory of {\it two} gauge fields, not restricted by any duality.

	Interestingly enough, the right way of ensuring duality, is not to impose a constraint, but to add a ``potential''
\[
	\big( F_{\mu\nu} ~-~ \wt G{}_{\mu\nu} \big)^2\,.
\]
	This seems unnatural at first.
	Nevertheless let us see what this leads to.

	The theory with the right coefficients is now
\begin{align}
%
\notag
	\cell    & ~~=~~    \frac{1}{4}\, \p_{[\mu} A_{\nu]}\, F_{\mu\nu}  ~~-~~  \frac{1}{8}\, F_{\mu\nu}^2  ~~+~~  j_\mu\, A^\mu  ~~+~~
	\\[2mm]
%
\label{action}
	&
		   ~~+~~    \frac{1}{4}\, \p_{[\mu} B_{\nu]}\, G_{\mu\nu}  ~~-~~  \frac{1}{8}\, G_{\mu\nu}^2  ~~+~~  k_\mu\, B^\mu  ~~+~~
	\\[2mm]
%
\notag
	&
	           ~~+~~    \frac{1}{16}\, \big( F_{\mu\nu} ~-~ \wt G{}_{\mu\nu} \big)^2\,.
\end{align}
	The origin of all fractional coefficients is the $ \frac{1}{4} $ in the usual kinetic term $ \frac{1}{4} F_{\mu\nu}^2 $ 
	in electrodynamics, which normalizes the degrees of freedom (in this sense, $ 4\, \cell $ looks a lot more plain).

	Equations for $ F_{\mu\nu} $ and $ G_{\mu\nu} $ immediately give the duality constraint
\beq
	\p_{[\mu} A_{\nu]}    ~~=~~    \frac{1}{2}\,\epsilon_{\mu\nu\rho\sigma}\, \p_{[\rho} B_{\sigma]}\,.
\eeq
	Specifically, their equations of motion are,
\begin{align}
%
\notag
	2\, \p_{[\mu} A_{\nu]}    & ~~=~~    F_{\mu\nu}  ~~+~~  \wt G{}_{\mu\nu}\,,
	\\[2mm]
%
\label{eq-motion}
	2\, \p_{[\mu} B_{\nu]}    & ~~=~~    G_{\mu\nu}  ~~+~~  \wt F{}_{\mu\nu}\,.
\end{align}
	Before substituting them back into the action, it is useful to re-arrange the Lagrangian \eqref{action} as follows,
\begin{align}
%
\notag
	\cell    & ~~=~~    \frac{1}{8}\, \big( \p_{[\mu} A_{\nu]} \big)^2  ~~-~~  
			    \frac{1}{8}\, \Big( \p_{[\mu} A_{\nu]} ~-~ F_{\mu\nu} \Big)^2  ~~+~~  j_\mu\, A^\mu  ~~+~~
	\\[2mm]
%
	&
		   ~~+~~    \frac{1}{8}\, \big( \p_{[\mu} B_{\nu]} \big)^2  ~~-~~
			    \frac{1}{8}\, \Big( \p_{[\mu} B_{\nu]} ~-~ G_{\mu\nu} \Big)^2  ~~+~~  k_\mu\, B^\mu  ~~+~~
	\\[2mm]
%
\notag
	&
	           ~~+~~    \frac{1}{16}\, \big( F_{\mu\nu} ~-~ \wt G{}_{\mu\nu} \big)^2\,.
\end{align}
	Now, when $ F_{\mu\nu} $ and $ G_{\mu\nu} $ are substituted, 
	all three bigger round brackets cancel,
	and all that is left is
\beq
\label{resolved}
	\cell    ~~=~~    \frac{1}{4}\, \big( \p_{[\mu} A_{\nu]} \big)^2  ~~+~~  j_\mu\, A^\mu    ~~+~~  k_\mu\, B^\mu\,,
\eeq
	with $ A_\mu $ and $ B_\mu $ constrained by duality.


\pagebreak
	Now, there is more than one way to derive Eq.~\eqref{resolved} --- either exclude $ F_{\mu\nu} $ first, then $ G_{\mu\nu} $,
	or exclude $ G_{\mu\nu} $ first and then $ F_{\mu\nu} $, or exclude them simultaneously as above. 
	The important point, however, is that the equations of motion derived from Eq.~\eqref{action} {\it do allow} 
	for $ A_\mu $ to be the sum of the fields of electric and magnetic sources --- that is, the total gauge potential.
	The action \eqref{resolved} ensures that it should be the only solution, up to gauge transformations of course.

	The theory \eqref{action} does not introduce constraints explicitly.
	At first glance, it does seem like action \eqref{action} is still nothing but a simple imposition of the constraint
\[
	\p_{[\mu} A_{\nu]}    ~~=~~    \frac{1}{2}\,\epsilon_{\mu\nu\rho\sigma}\, \p_{[\rho} B_{\sigma]}
\]
	via a tensor Lagrange multiplier --- the thing that we wanted to avoid in the first place.
	
	In fact it is more than just that.
	To see that, we need to convert the theory to the Hamiltonian form.
	One of the advantages of the first order formalism \eqref{QED} for QED or Yang--Mills, is that it explicitly introduces 
	the momentum variables for the gauge field: $ F_{0j} $ is precisely the momentum for $ A_j $.
	As for the remaining $ F_{ik} $  --- we do not need them, 
	and they can simply be replaced with the magnetic field $ [ \nabla A ] $.
	Then, for example, the ordinary QED action \eqref{QED} is already written in terms of the canonical variables and their momenta,
	and both the Hamiltonian and the gauge constraint are just read off it:
\[
	\cell    ~~=~~    \pi_j\, \dot{q}{}_j  ~~-~~  H(\pi,\, q)  ~~+~~  A_0\, \phi(\pi,\, q)\,.
\]

	The same is true about action \eqref{action}, it is already given in terms the canonical variables,
	only now there are two sets of variables and momenta.
	Again, only the magnetic components $ F_{ik} $ and $ G_{ik} $ need to be resolved. 
	Once this is done, the theory has the form 
\[
	\cell    ~~=~~    \pi_j\, \dot{q}{}_j  ~~+~~ \sigma_j\, \dot{r}{}_j  ~~-~~  H(\pi,\, q\,, \sigma\,, r)  
	~~+~~  A_0\, \phi(\pi,\, q)  ~~+~~  B_0\, \phi(\sigma,\, r)\,.
\]
	This form of the Lagrangian (Hamiltonian) is sufficient to determine the dynamics classically, and 
	so we are inclined to think that no additional constraints are introduced. 

	Quantization of this theory remains to be considered yet


\end{document}
